\documentclass{article}
\usepackage[utf8]{inputenc}
\usepackage[spanish]{babel}
\usepackage{listings}
\usepackage{graphicx}
\graphicspath{ {images/} }
\usepackage{cite}

\begin{document}

\begin{titlepage}
    \begin{center}
        \vspace*{1cm}
            
        \Huge
        \textbf{La importancia de dar buenas instrucciones}
            
        \vspace{0.5cm}
        \LARGE
        Actividad 1
            
        \vspace{1.5cm}
            
        \textbf{Manuela Gutiérrez Rodríguez}
            
        \vfill
            
        \vspace{0.8cm}
            
        \Large
        Despartamento de Ingeniería Electrónica y Telecomunicaciones\\
        Universidad de Antioquia\\
        Medellín\\
        9 de Marzo de 2021
            
    \end{center}
\end{titlepage}

\tableofcontents
\newpage
\section{Introducción}\label{intro}
La evolución tecnológica en la que nos encontramos es de vital importancia para el avance de la humanidad, por esta razón como estudiante de ingeniería de telecomunicaciones soy consciente de que es necesario tener conocimiento y control sobre la programación.

\vspace{5mm}

Este ejercicio es una forma de autoevaluación para ver como es mi dsempeño a la hora de querer dar instrucciones a un ser humano o a un computador y que estos sean capaces de materializar la idea que les estoy brindando.

\section{Contenido} \label{contenido}

A continuación se encontrarán las indicaciones tal cual como se le entregaron a los participantes:

\vspace{5mm}

\textbf{Advertencias:}

a) Únicamente puede utilizar una mano.

b) No puede preguntar nada a nadie, debe hacerlo según su interpretación.

\vspace{5mm}

\textbf{Instrucciones}

\vspace{5mm}

1. Busque una hoja de papel completamente blanca y lisa.

2. Tome la hoja por uno de sus extremos con mucho cuidado.

3. Busque una superficie plana, lisa y que sea horizontal. (En lo posible que sea una mesa estable)

4. Ubique la hoja cuidadosamente sobre la superficie. Asegúrese que no haya ningún objeto sobre el espacio en el cual va a posicionar la hoja.

5. Busque dos trajetas que sean del mismo tamaño y peso aproximadamente.

6. Ubique una de las tarjetas sobre otra, de modo que quede cara con cara, además que los lados cortos coincidan y los largos también.

7. A continuación, ubique uno de los lados cortos del conjunto de tarjetas, con su dedo índice y pulgar posicionelos en los extremos del lado corto que eligió y sostengalas.

8. Ubique el extremo contrario de las tarjetas del cual las está sosteniendo y póngalo sobre la hoja, en el centro de ésta.

9. Con el dedo de la mitad de la mano con la que sostiene las tarjetas desplace una de ellas hasta formar un ángulo interno de 30° aproximadamente.

10. Por último, ubique la parte de abajo de la tarjeta que desplazó sobre la hoja y suelte cuidadosamente ambas tarjetas. Estas le deberían quedar en forma de píramide.

Opcional: Si falla, puede volver a intentarlo desde el paso 6.

\section{Conclusiones} \label{conclusiones}

A partir de este ejercicio pude notar un gran error que muchas veces cometo sin darme cuenta y es creer que los demás piensan y ven las cosas al igual que yo. A pesar de que este ejercicio se realizó con personas se puede evidenciar fácilmente que para que una instrucción sea procesada y ejecutada correctamente se debe ser lo más claro y preciso posible. Llevando esto a la practicidad, el procesador interpreta tal cual las instrucciones que recibe, por ello éstas deben ser lo menos ambiguas posible. 

\vspace{5mm}

Por otra parte, para mejorar la eficiencia a la hora de trabajar, es necesario comprender el problema y darle solución antes de empezar a programar, de esta forma se evitan muchos errores y se ahorra tiempo, una vez que el análisis esté realizado ya se avanzó en un gran porcentaje del trabajo.

\end{document}
